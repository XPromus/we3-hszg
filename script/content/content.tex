\documentclass[../main.tex]{subfiles}
\begin{document}

\subfile{abstract}
\section{Einführung}
\cite{sansom1958history}
\gls{latex}
\Gls{latex}
\acrshort{gcd}
\lipsum[1-2]

\newpage
\section{Farbboxen}
\begin{tcolorbox}[colback=orange!5!white,colframe=orange!75!black,title=\textbf{Farbboxe Orange}]
Das ist eine Farbbox. Sie ist eine Box und enthält Farbe.
\end{tcolorbox}

\begin{tcolorbox}[colback=red!5!white,colframe=red!75!black,title=\textbf{Farbboxe Rot}]
Das ist eine Farbbox. Sie ist eine Box und enthält Farbe.
\end{tcolorbox}

\begin{tcolorbox}[colback=blue!5!white,colframe=blue!75!black,title=\textbf{Farbboxe Blau}]
Das ist eine Farbbox. Sie ist eine Box und enthält Farbe.
\end{tcolorbox}

\newpage
\section{Plan}
\begin{itemize}
    \item User Storys als Zielsetzung für das Projekt
\end{itemize}

\section{Technologien}
\subsection{Frontend}
\begin{itemize}
    \item Svelte/SvelteKit
    \item Vue
    \item React
    \item Angular
    \item Astro
    \item Vanilla
    \item Rails
    \item Laravell
    \item Symphony
    \item Vite/Webpack
    \item NodeJS/Bun/Deno/Yarn
    \item TailwindCSS
    \item PRIME
    \item How to Debug web apps
\end{itemize}
\subsection{Backend}
\begin{itemize}
    \item Tomcat Servelet zu Spring Beans (Aufbau)
    \item Spring
    \item Micronaut
    \item Quarkus
    \item Kotlin
    \item Gradle
    \item Debug: Conditional Breakpoints, IntelliJ Evaluate Expression
\end{itemize}
\subsection{Database}
\begin{itemize}
    \item PostgreSQL
\end{itemize}
\subsection{Testing}
\begin{itemize}
    \item Mal Ansprechen
    \item Cypress, Playwright
\end{itemize}
\subsection{Devops}
\begin{itemize}
    \item Docker
    \item Podman
\end{itemize}

\section{Anforderungen}
\begin{itemize}
    \item Jeder Endpunkt der API soll dokumentiert sein in OpenAPI
    \item Verteidigung der Belege durch Präsentation
    \item Repository zur Bewertung schicken mit Anleitung zum Deployment 
    \item
\end{itemize}

\section{Ablauf \& Planung}
\begin{itemize}
    \item Ziel des Moduls: Projektarbeit mit einem Beleg und einer Präsentation als finales Ziel
    \item Projekt sollte einen Großteil der Aspekte der Web Entwicklung abdecken
    \item Basis für das Projekt: User-Stories, die das Projekt leiten sollen
    \item Gruppenarbeit möglich
    \item Beleg mit Code als Abgabe am Ende des Semesters
\end{itemize}

\subsection{Anforderungen an das Projekt}
\begin{itemize}
    \item Datenbank, Backend, Frontend mit Container Deployment (Docker, Podman, ...)
    \item Dokumentation der REST API Endpunkte mit OpenAPI o.ä.
    \item Einige Tests in Front- und Backend. Komplettes Test coverage wird nicht vorausgesetzt
    \item Einreichung des Repositories (ZIP, Link zu GitHub oder andere VCS)
\end{itemize}

\subsection{Anforderungen an den Beleg}
\begin{itemize}
    \item Seitenanzahl nicht festgelegt. Bewegt sich wahrscheinlich um 20 Seiten, wird aber nicht vorausgesetzt
    \item Beschreibung, wie Anforderungen aus den User Stories umgesetzt wurden
    \item Umsetzung beschreiben
    \item Gründe für Entscheidungen bei der Entwicklung darstellen
    \item Dokumentation der einzelnen Software Bestandteile 
\end{itemize}

\subsection{Anforderungen an die Präsentation}
\begin{itemize}
    \item Demonstration des finalen Produkts
    \item Vorstellung der Umsetzung
    \item Kurzes Zeigen von ausgewählten Programmbestandteilen, die als wichtig angesehen werden
\end{itemize}

\section{User Stories}
Kernpunkte einer User Story \cite{anatomyOfAUserStory}:
\begin{itemize}
    \item Wer ist der User?
    \item Was will der User machen?
    \item Warum will der User das machen?
    \item Weitere Informationen sind optional
\end{itemize}

\subsection{Template \cite{anatomyOfAUserStory}}
AS A \{user|persona|system\} \\
INSTEAD OF \{current condition\} \\
I WANT TO \{action\} IN \{mode\} TIME | IN \{differentiating performance units\} TO \{utility performance units\} \\
SO THAT \{value of justification\} \\
NO LATER THAN \{best by date\}

\subsection{User Stories für das finale Projekt}


\end{document}